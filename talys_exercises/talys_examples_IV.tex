%\documentclass[handout]{beamer}

\documentclass{beamer}

%\usepackage{pgfpages}
%\usepackage{handoutWithNotes}
%\pgfpagesuselayout{4 on 1 with notes}[a4paper,border shrink=5mm]

\usepackage{etex}
 
 
%\usepackage[ngerman, american, british]{babel} 
\usepackage[T1]{fontenc}
\usepackage[utf8]{inputenc} %%% latin input
%\DeclareUnicodeCharacter{00A0}{ }

%\usepackage{default}
\usetheme{Warsaw}
%\usepackage{authblk}
\usepackage{multicol}
%\usepackage[style=numeric,backend=biber]{biblatex}
% %\usepackage[style=numeric]{biblatex}
%\addbibresource{../bib.bib}
%\AtEveryCitekey{%
%\clearfield{title}%
%\clearfield{pagetotal}%
%}
% Make bracketized
%\renewcommand*{\thefootnote}{[\arabic{footnote}]}
%\makeatletter
% Remove superscript for footnotemark
%\def\@makefnmark{\hbox{{\normalfont\@thefnmark}}}
% Allow space to precede the footnote
%\usepackage{etoolbox}
%\patchcmd{\blx@mkbibfootnote}{\unspace}{}{}
%\makeatother
\setbeamertemplate{navigation symbols}{}%remove navigation symbols


\usepackage{textpos}
\usepackage[percent]{overpic}
\usepackage{adjustbox}

\newcommand{\PreLim}
{\begin{tikzpicture}
% \node [opacity=0.1,rotate around={45:(0,0)}] (0,0) {\scalebox{2.0}{\textcolor{green}{Preliminary}}};
\end{tikzpicture}}
\newcommand{\PreLimP}{\put (50,0) {\PreLim}}
  
%\usepackage[printwatermark]{xwatermark}
%\newwatermark*[allpages,color=red!50,angle=45,scale=2,xpos=0,ypos=0]{DRAFT}

%\usepackage{struktex} %%% create structograms
\usepackage{booktabs} %%% scientific publication style tables
\usepackage{empheq}   %%% for boxed equations and alike
\usepackage{xcolor}   %%% for colored text and formulae
\usepackage{amsmath}  %%% math symbols needed
\usepackage{amsfonts} %%% math fonts needed
\usepackage{amssymb}  %%% maths symbols needed 
\usepackage{amsthm}   %%% theorem environments
%\usepackage{dsfont}   %%% for blackboard numbers and characters 
%\usepackage{mathrsfs} %%% fonts when speaking e.g. about Lagrangian density
%\usepackage{mathdots/mathdots}
\usepackage{pdfpages}
\usepackage{multirow} %%% merge rows in a table

% Default graphics paths
\graphicspath{{figs/}}

%use of ps ticks
% \usepackage[usenames,dvipsnames]{pstricks}
% \usepackage{epsfig}
% \usepackage{pst-grad} % For gradients
% \usepackage{pst-plot} % For axes
\usepackage{pgf,tikz}
\usetikzlibrary{shapes,arrows,positioning,calc}
\usetikzlibrary{tikzmark} % remember position for later

\ifx\du\undefined
  \newlength{\du}
\fi
\setlength{\du}{15\unitlength}

%\usepackage{fancybox} %%% more types of boxes 

%\usepackage[pdftex]{graphicx}       %%% CUSTOMIZATION: include all pictures
%\usepackage[draft]{graphicx} %%% choose draft option for faster compiling without pictures
%\graphicspath{{Figures/}}      %%% all pictures will reside there and in subdirs


%%%%%%%%%%%%%%%%%%%%%%%%%%%%%%%%%%%%%%%%%%%%%%%%%%
%%% acronym and notation section
%%% comment/delete this section, if you do not want to use these
%\usepackage[toc,style=list,acronym=true]{glossary} %%% this package has to go AFTER the hyperref package!
%\setacronymnamefmt{\gloshort}
%\setacronymdescfmt{\glolong}
%\input{Misc/acronyms-definitions.tex}
%\makeacronym
%
%\newglossarytype{notation}{not}{ntn}[style=list]
%\newcommand{\notationname}{Notation}
%\setglossary{type=notation,glsnumformat=ignore,glodelim={.}}
%\makenotation
%%% end of acronuyms and notation
%%%%%%%%%%%%%%%%%%%%%%%%%%%%%%%%%%%%%%%%%%%%%%%%%%

\newcommand\FourQuad[4]{%
    \begin{minipage}[b][.35\textheight][t]{.47\textwidth}#1\end{minipage}\vrule  \hfill%
    \begin{minipage}[b][.35\textheight][t]{.47\textwidth}#2\end{minipage} \\[0.25em]
    \hrule \hspace{0.001cm} \\[0.25em]
    \begin{minipage}[b][.35\textheight][t]{.47\textwidth}#3\end{minipage}\vrule \hfill
    \begin{minipage}[b][.35\textheight][t]{.47\textwidth}#4\end{minipage}%
    }

%%%%%%%%%%%%%%%%%%%%%%%%%%%%%%%%%%%%%%%%%%%%%%%%%%
%%% Selection of the fonts!! 
\usepackage[T1]{fontenc}
%\usepackage{helvet}
%%%%%%%%%%%%%%%%%%%%%%%%%%%%%%%%%%%%%%%%%%%%%%%%%%

\newenvironment<>{varblock}[2][.9\textwidth]{%
  \setlength{\textwidth}{#1}
  \begin{actionenv}#3%
    \def\insertblocktitle{#2}%
    \par%
    \usebeamertemplate{block begin}}
  {\par%
    \usebeamertemplate{block end}%
  \end{actionenv}}


%\usepackage{tikz}
%\usetikzlibrary{arrows,shapes}
% For every picture that defines or uses external nodes, you'll have to
% apply the 'remember picture' style. To avoid some typing, we'll apply
% the style to all pictures.
\tikzstyle{every picture}+=[remember picture]

% By default all math in TikZ nodes are set in inline mode. Change this to
% displaystyle so that we don't get small fractions.
\everymath{\displaystyle}

\setbeamercovered{transparent}



\title[Talys examples -- 4 (Wednesday)] % (optional, only for long titles)
{Introduction into the usage of TALYS -- VI}
\subtitle{Oslo-Berkeley-Stellenbosch-iThemba Summer School 2019}

\author[F. Zeiser]{}

%\date[12th September 2016] % (optional)
%{ND2016, Brugge}
%\subject{Computer Science}r}



%%%%%%%%%%%%%%%
%\usepackage[left=1.00in, right=1.00in, top=1.00in, bottom=1.00in]{geometry}
\usepackage{xpatch}
\usepackage{tcolorbox}
\tcbuselibrary{minted,skins,listings}

\tcbset{
        enhanced,
        boxrule=0.0pt,
        %fonttitle=\bfseries
       }

\newtcblisting{bashcodebg}[1][]{
    listing engine=minted,
    colback=bashcodebg,
    colframe=black!70,
    listing only,
    minted style=colorful,
    minted language=bash,
    minted options={linenos=true,numbersep=3mm,texcl=true,breaklines,#1},
    left=5mm,enhanced,
    overlay={\begin{tcbclipinterior}\fill[black!25] (frame.south west)
            rectangle ([xshift=5mm]frame.north west);\end{tcbclipinterior}}
}

\newtcblisting{bashcodebg2}[1][]{
    listing engine=minted,
    colback=bashcodebg,
    colframe=black!70,
    listing only,
    minted style=colorful,
    minted language=bash,
    minted options={linenos=false,numbersep=3mm,texcl=false,#1,xrightmargin=0.5\textwidth},
    left=0mm,enhanced,
    %overlay={\begin{tcbclipinterior}\fill[black!25] (frame.south west)
    %        rectangle (frame.north west);\end{tcbclipinterior}}
}

\definecolor{bashcodebg}{rgb}{0.85,0.85,0.85}

%%%%%%

\newlength{\mintedfboxsep}
\setlength{\mintedfboxsep}{0.2pt}

\newmintinline{python}{python3, framesep=0.1pt,bgcolor=bashcodebg}

\makeatletter
\newlength{\fboxrsep}
\setlength{\fboxrsep}{\fboxsep}

\newlength{\fboxlsep}
\setlength{\fboxlsep}{\fboxsep}

\newlength{\fboxtsep}
\setlength{\fboxtsep}{\fboxsep}

\newlength{\fboxbsep}
\setlength{\fboxbsep}{\mintedfboxsep}

\xpatchcmd{\minted@inputpyg@inline}{%
  \colorbox%
}{%
  \long\def\color@b@x##1##2##3%
  {\leavevmode
    \setbox\z@\hbox{\kern\fboxlsep{\set@color##3}\kern\fboxrsep}%
    \dimen@\ht\z@\advance\dimen@\fboxtsep\ht\z@\dimen@
    \dimen@\dp\z@\advance\dimen@\fboxbsep\dp\z@\dimen@
    {##1{##2\color@block{\wd\z@}{\ht\z@}{\dp\z@}\box\z@}}}%
  \colorbox%
}{\typeout{Success}}{\typeout{Failure}}
\makeatother

%%%%%%%



%%%%%%%%%

\begin{document}

\maketitle

%%%%%%%%%%%%%%%%%%%%%

\begin{frame}{Task}
\tiny
You want to calculate the cross-sections for 191Os(n,$\gamma$) and 191Os(n,tot). With a 15days half-life it's difficult to make a target out of 191Os, thus there is no tabulated data on $D_0$ and $\langle \Gamma_\gamma \rangle$ in Talys.
\begin{itemize}
	\item Assume that you have obtained a $D_0$ of 4.13(20) eV and $\langle \Gamma_\gamma \rangle$ of 0.049(20) eV from your magic systematics\footnote{\tiny These are arb. chosen here!}. Renormalize the NLD and $\gamma$SF to match these values.
	\item For the nld and gsf, choose (a) an empirical and (b) a semi-microscopic model.
	\item Compare the default and renormalized NLD(s) and $\gamma$SF(s) and their impact in the cross-sections. Choose sensible regions for the incident particle energies and the cross-sections.
	\item Pick one of the above options and study the effect of the neutron-nucleus OMP on both cross-sections by modifying the depth of the real and imaginary potentials. \textit{Before} you start: What do you expect? \\ Visualize the results and explain the behavior.
\end{itemize}

Tips:
\begin{itemize}
	\item To modify NLDs you may modify/use the keywords \pythoninline{a},  \pythoninline{T}, and  \pythoninline{E0}, or for tabulated NLDs: \pythoninline{ctable} and  \pythoninline{ptable}
	\item For most parameters \pythoninline{p} here a scaling keyword \pythoninline{padjust} exists. It will work like $p_{new} = p *  padjust$
	\item To modify the $\gamma$SF you may modify/use the keywords like \pythoninline{egr} (...), or for tabulated gsfs: \pythoninline{etable} and  \pythoninline{ftable}
	\item To renormalize the $\gamma$SF $f$ you can alternatively scale it by $f_\mathrm{new} = f_\mathrm{org} * \pythoninline{gnorm}$.\\
	Attention: If exp. data on $\langle \Gamma_\gamma \rangle$is available, and you don't specify \pythoninline{gnorm 1.}, the $\gamma$SF is automatically scaled to match $\langle \Gamma_\gamma \rangle$.
	\item The real and imaginary well-depth can be scaled by  \pythoninline{v1adjust} and  \pythoninline{w1adjust}.
	\item Annotated output file on the next slides (reminder)
\end{itemize}




\end{frame}


%%%%%%%%%%%%%%%%



\begin{frame}[fragile]{Information on $a$ and $D_0$}
	
	Investigation of the output file: Default; output with \pythoninline{outdensity y}
	
	\begin{adjustbox}{minipage=1.2\textwidth,margin=0pt \smallskipamount,center}
		% WORKAROUND
		\begingroup % WORKAROUND
		\catcode`\!=\active   % WORKAROUND
		\def!#1!{\setlength{\fboxsep}{0pt}\colorbox{yellow!60}{#1}}  % WORKAROUND
		\def¤#1¤{\setlength{\fboxsep}{0pt}\colorbox{black!30}{[...]}}  % WORKAROUND
		%\defø#1ø{\colorbox{green!06}{\color{black!40} #1}}  % WORKAROUND
		\defø#1ø{\setlength{\fboxsep}{0pt}\colorbox{green!06}{\color{black!40} #1}}
		\begin{tiny}
			\begin{bashcodebg}[escapeinside=||]
|¤¤|
Level density parameters for Z= 76 N=116 |!(192Os)!| 

Model: Gilbert-Cameron          
Collective enhancement: no

a(Sn)           :  22.22718									  |øa for the phenom. modelø|
Experimental D0 :              0.00 eV +-         0.00000     |øExperimental= RIPL-2ø| 
Theoretical D0  :              3.45 eV                        |øTheor.= TALYS calc ø|
Asymptotic a    :  22.70807
Damping gamma   :   0.07507
Pairing energy  :   1.73205
Shell correction:  -0.34826
Last disc. level:        10
Nlow            :         4
Ntop            :        17
Matching Ex     :   6.03773
Temperature     :   0.51816
E0              :   0.00483
Adj. pair shift :   0.00000
Discrete sigma  :   3.60794
Sigma (Sn)      :   6.67144
|¤¤|
Ex     a    sigma   total   JP=  0.0  JP=  1.0  JP=  2.0  JP=  3.0  |¤¤|

0.25 22.114  3.608 1.549E+00 5.892E-02 1.637E-01 2.340E-01 2.|¤¤|
|¤¤|
			\end{bashcodebg}
		\end{tiny}
		\endgroup
	\end{adjustbox}
	
\end{frame}






%%%%%%%%%%%%%%%


\begin{frame}[fragile]{Information on D0 and $\gamma$SF}
	
	Investigation of the output file: Default; output with \pythoninline{outgamma y}
	
	\begin{adjustbox}{minipage=1.2\textwidth,margin=0pt \smallskipamount,center}
		% WORKAROUND
		\begingroup % WORKAROUND
		\catcode`\!=\active   % WORKAROUND
		\def!#1!{\setlength{\fboxsep}{0pt}\colorbox{yellow!60}{#1}}  % WORKAROUND
		\def¤#1¤{\setlength{\fboxsep}{0pt}\colorbox{black!30}{[...]}}  % WORKAROUND
		%\defø#1ø{\colorbox{green!06}{\color{black!40} #1}}  % WORKAROUND
		\defø#1ø{\setlength{\fboxsep}{0pt}\colorbox{green!06}{\color{black!40} #1}}
		\begin{tiny}
			\begin{bashcodebg}[escapeinside=||]
|¤¤|
########## GAMMA STRENGTH FUNCTIONS, TRANSMISSION COEFFICIENTS AND CROSS SECTIONS ##########
Gamma-ray information for Z= 76 N=116 (192Os) 

S-wave strength function parameters:
 |øExperimental= RIPL-2ø|                                 |øTheor.= TALYS calc; doesn't regard gnorm ø|
Exp. total radiative width=   0.08450 eV +/- 0.00000 Theor. total radiative width=        0.02039 eV
Exp. D0                   =      0.00 eV +/-    0.00 Theor. D0                   =        3.44583 eV
Theor. S-wave strength f. =  59.17388E-4
Normalization factor      =   1.00000

Gamma-ray strength function model for E1: Kopecky-Uhl              

Gamma-ray strength function model for M1: RIPL-2                   

Normalized gamma-ray strength functions and transmission coefficients for l= 1

Giant resonance parameters :

sigma0(M1) =   0.860       sigma0(E1) = 206.000 and  389.000    PR: sigma0(M1) =   0.000       sigma0(E1) =   0.00
|¤¤|

E       f(M1)        f(E1)        T(M1)        T(E1)

0.001  0.00000E+00  1.02591E-08  0.00000E+00  2.85769E-16
0.002  9.35634E-13  1.02584E-08  4.70301E-20  2.28630E-15
|¤¤|
			\end{bashcodebg}
		\end{tiny}
		\endgroup
	\end{adjustbox}
	
\end{frame}

%%%%%%%%%%%%%%%%%

\end{document}